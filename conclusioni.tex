\section{Conclusioni: errori sistematici}
Andremo in seguito ad elencare ed analizzare le possibili cause di errore sistematico che si incontrano nella deteminazione dello zero assoluto $T_0$.

\subsection{Volume di gas non termalizzato con il bagno termico}
Una possibile causa di errore sistematico è la possibile presenza di gas non termalizzato con il bagno termico all'interno dell'ampolla.
È infatti possibile che il gas dell'ampolla presenti una parte di volume $V$ termalizzato con il bagno termico ed una parte di volume $v$ non termalizzato con il bagno termico e che si trova quindi a temperatura ambiente.
Considerando $n$ moli di gas nell'ampolla, l'equazione di stato diventa:
\begin{equation}
\label{eq:nR}
n R = P_0 \left[\frac{V}{T_0} + \frac{v}{T_a}\right] = P \left[\frac{V}{T} + \frac{v}{T_a}\right]
\end{equation}
dove $T_a$ rappresenta la temperatura ambiente.
Otteniamo quindi:
\begin{equation}
\frac{P_0 V}{T_0} \left[1 + v T_0 V T_a\right]= \frac{P V}{T} \left[1 + \frac{v T}{V T_a}\right]
\end{equation}
La relazione tra temperatura $T$  e pressione assoluta $P$ va quindi corretta, ottenendo quindi:
\begin{equation}
P = \left(\frac{P_0}{T_0}\right) \frac{1 + \left(\frac{v T_0}{V T_a} \right)}{1 + \left(\frac{v T}{V T_a}\right)} T = P_0 \frac{1 + \left(\frac {v T_0}{V T_a}\right)}{1 + \left(\frac{v T}{V T_a}\right)} \left(1 + \alpha T \right)
\end{equation}
Al crescere della temperatura quindi la pedenza della funzione $P(T)$ diminuisce.
Ad ogni modo, anche se il volume del gas dell'ampolla $v$ fosse effettivamente tutto termalizzato, nel caso reale andrebbe comunque applicata una correzione sulla pressione del tipo:
\begin{equation}
P_{mis} = \frac{n R T}{V + \frac{v T}{T_a}}
\end{equation}

\subsection{Dilatazione termica del vetro}
Un'altra possibile caus di errore sistematico è dovuto alla dilatazione termica del vetro dell'ampolla che contine il gas grazie al quale prendiamo le nostre misurazioni.
Questa ampolla contiene quindi un certo volume di gas che però, contrariamente a quanto richiesto dalla  esperienza, non è costante ma varia con la temperatura seguendo una legge del tipo:
\begin{equation}
\label{eq:V}
V = V_0 (1 + \gamma \theta)
\end{equation}
dove $V_0$ è il volume a $0^{\degree}$ C e $\gamma$ è un coefficiente di espansione di volume dipendente dal tipo di vetro.
L'equazione \eqref{eq:nR} deve quindi tenere conto della distinzione tra volume e temperature, ovvero:
\begin{equation}
n R = P_0 \left[\frac{V_0}{T_0} + \frac{v}{T_a}\right] = P \left[\frac{V}{T} + \frac{v}{T_a}\right]
\end{equation}
Tenendo conto di \eqref{eq:V} otteniamo infine:
\begin{equation}
P = P_0 \frac{1}{1 + \gamma \theta} \frac{1 + \left(\frac{v T_0}{V_0 T_a} \right)}{1 + \left(\frac{v T}{V_0 T_a} \right)}(1 + \alpha \theta)
\end{equation}

\subsection{Influenza dell'umidità}
