\section{Introduzione}
\subsection{Preparazione dell'apparato sperimentale e strategie di misura}

Per questa esperienza è d'obbligo prestare una particolare attenzione alla costruzione dell'apparato sperimentale, in quanto anche piccole dimenticanze possono causare variazioni nelle misurazioni.
Si proceda quindi con la preparazione dell'apparato: verssare un pò d'acqua nel contenitore, posarlo sopra l'agitatore magnetico ed inserendo l'ancoretta magnetico all'interno di esso, controllare che la rotazione di quest'ultima.
Questa prima accortezza è necessaria in quanto l'acqua diventerà il bagno termico per tutta la durata dell'esperimenti e quindi il mescolamento dell'agitatoree termico è una condizione necessaria per l'omogeneità dell'acqua che andrebbe, in sua assenza, a stratificarsi in base alla sua temperatura.
Controllare quindi che l'interno della bottiglia contenente il gas non sia bagnato e, tenendo aperti i suoi rubinetti, fissarla stabilmente in modo da restitere alla spinta idrostatica ed in modo da permettere il funzionamento dell'agitatore magnetico.
Collegre un ramo del manometro ad uno dei rubinetti della bottiglia ed, aiutandosi con la siringa di plastica, versare acqua nel manometro tenendo i due rubinetti alla stessa altezza e stando particolarmente attenti ad evitare la formazione di bolle d'aria.
Riempire il manometro di acqua il più possibile, ovvero prestare attenzione che l'acqua raggiunga il più alto livello possibile non superando la parte rigida trasparente, grazie alla quale si prenderanno le misure sulla scala millimetrata posta accanto.
Verificare che neanche una goccia d'acqua sia entrata nella bottiglia.

\subsection{Termalizzazione iniziale dell'aria}
Procedere quindi con la termalizzazione dell'aria.
Preparare il bagno termico intorno alla bottiglia e controllare l'uniformità della temperatura, garantita dall'agitatore termico.
Chiudere il rubinetto isolando il gas all'interno della bottiglia dall'ambiente esterno e garantendo quindi che il numero di moli di gas rimanga costante.
Scegliamo di termalizzare inizialmente intorno ai $0^{\degree}$ per poi salire fino ad una tmperatura di circa $20^{\degree}$ e quindi ridiscendere fino al valore iniziale.
Scegliendo questa procedura di misurazione la bottiglia si troverà sempre in sovrapressione rispetto alla pressione atmosferica: in questo caso allora il manometro dovrà essere montato sul tavolo con il lato libero in grado di salire di circa 80 cm sopra la posizione iniziale (si consiglia quindi di scegliere un valore iniziale che non superi i 20 cm).
Scegliendo di termalizzare il sitema a $0^{\degree}$, la presenza di ghiaccio nel bagno dovrà essere presente in maniera predominante rispetto all'acqua: la coesistenza di queste due fasi assicura che la temperatura sia circa uguale allo zero termico.
Immergere quindi la bottiglia nel bagno termico fino al tappo, prestando attenzione che non raggiunga i giunti di connessione dei rubinetti.
Una volta raggiunto l'equilibrio termico alla temperatura iniziale, aprire il rubinetto e richiuderlo: in questo modo la pressione del gas all'interno della bottiglia viene uniformata con la pressione atmosferica $P_{A}$ ed il volume e le moli del gas sono costanti. 
Si potrà procedere con l'esperienza.


\section{Misura di pressione al variare della temperatura}
