\section{Introduzione}
\subsection{Preparazione dell'apparato sperimentale e strategie di misura}

Per questa esperienza è d'obbligo prestare una particolare attenzione alla costruzione dell'apparato sperimentale, in quanto anche piccole dimenticanze possono causare variazioni nelle misurazioni.
Si proceda quindi con la preparazione dell'apparato: verssare un pò d'acqua nel contenitore, posarlo sopra l'agitatore magnetico ed inserendo l'ancoretta magnetico all'interno di esso, controllare che la rotazione di quest'ultima.
Questa prima accortezza è necessaria in quanto l'acqua diventerà il bagno termico per tutta la durata dell'esperimenti e quindi il mescolamento dell'agitatoree termico è una condizione necessaria per l'omogeneità dell'acqua che andrebbe, in sua assenza, a stratificarsi in base alla sua temperatura.
Controllare quindi che l'interno della bottiglia contenente il gas non sia bagnato e, tenendo aperti i suoi rubinetti, fissarla stabilmente in modo da restitere alla spinta idrostatica ed in modo da permettere il funzionamento dell'agitatore magnetico.
Collegre un ramo del manometro ad uno dei rubinetti della bottiglia ed, aiutandosi con la siringa di plastica, versare acqua nel manometro tenendo i due rubinetti alla stessa altezza e stando particolarmente attenti ad evitare la formazione di bolle d'aria.
Riempire il manometro di acqua il più possibile, ovvero prestare attenzione che l'acqua raggiunga il più alto livello possibile non superando la parte rigida trasparente, grazie alla quale si prenderanno le misure sulla scala millimetrata posta accanto.
Verificare che neanche una goccia d'acqua sia entrata nella bottiglia.

\subsection{Termalizzazione iniziale dell'aria}
Procedere quindi con la termalizzazione dell'aria.
Preparare il bagno termico intorno alla bottiglia e controllare l'uniformità della temperatura, garantita dall'agitatore termico.
Chiudere il rubinetto isolando il gas all'interno della bottiglia dall'ambiente esterno e garantendo quindi che il numero di moli di gas rimanga costante.
Scegliamo di termalizzare inizialmente intorno ai $0^{\degree}$ per poi salire fino ad una temperatura di circa $20^{\degree}$ e quindi ridiscendere fino al valore iniziale.
Scegliendo questa procedura di misurazione la bottiglia si troverà sempre in sovrapressione rispetto alla pressione atmosferica: in questo caso allora il manometro dovrà essere montato sul tavolo con il lato libero in grado di salire di circa 80 cm sopra la posizione iniziale (si consiglia quindi di scegliere un valore iniziale che non superi i 20 cm).
Scegliendo di termalizzare il sitema a $0^{\degree}$, la presenza di ghiaccio nel bagno dovrà essere presente in maniera predominante rispetto all'acqua: la coesistenza di queste due fasi assicura che la temperatura sia circa uguale allo zero termico.
Immergere quindi la bottiglia nel bagno termico fino al tappo, prestando attenzione che non raggiunga i giunti di connessione dei rubinetti.
Una volta raggiunto l'equilibrio termico alla temperatura iniziale, aprire il rubinetto e richiuderlo: in questo modo la pressione del gas all'interno della bottiglia viene uniformata con la pressione atmosferica $P_{A}$ ed il volume e le moli del gas sono costanti. 
Procediamo poi con le misurazioni.\\
Da aggiungere paragrafo 9.4.3 scheda di laboratorio


\section{Analisi dati}
\subsection{Pressione e temperatura}

\begin{table}[H]
\centering

	\begin{subtable}{.5\textwidth}
		\centering
		\begin{tabular}{|c|c|} \hline
			\textbf{h {[cm]} } & \textbf{T {[\degree C]} }  \\ \hline
			130.5 & 2.306  \\ \hline
			130.5 & 2.306  \\ \hline
			130.5 & 2.306  \\ \hline
			130.5 & 2.306  \\ \hline
			130.5 & 2.306  \\ \hline
			130.5 & 2.306  \\ \hline
			130.5 & 2.306  \\ \hline
			130.5 & 2.306  \\ \hline
			130.5 & 2.306  \\ \hline
			130.5 & 2.306  \\ \hline
			130.5 & 2.306  \\ \hline
		\end{tabular}
		\caption{Aumentando la temperatura}
	\end{subtable}%
	\begin{subtable}{.5\textwidth}
	\centering
	\begin{tabular}{|c|c|} \hline
		\textbf{h {[cm]} } & \textbf{T {[\degree C]} }  \\ \hline
		130.5 & 2.306  \\ \hline
		130.5 & 2.306  \\ \hline
		130.5 & 2.306  \\ \hline
		130.5 & 2.306  \\ \hline
		130.5 & 2.306  \\ \hline
		130.5 & 2.306  \\ \hline
		130.5 & 2.306  \\ \hline
		130.5 & 2.306  \\ \hline
		130.5 & 2.306  \\ \hline
		130.5 & 2.306  \\ \hline
		130.5 & 2.306  \\ \hline
	\end{tabular}
	\caption{Diminuendoo la temperatura}
\end{subtable}

\caption{Dislivelo in funzione della temperatura}
\end{table}
In tabella 1 ci sono i dati che abbiamo raccolto durante la nostra esperienza, sono divisi i due tabelle, una relative alle misure prese facendo aumentare la temperatura mentre la seconda relativa alle misure prese facendo diminuire la temperatura. Questi dati sono graficati nella Figura 1. In tale figura abbiamo scelto di rappresentare i punti relativi alla Tabella 1.a in colore rosso, mentre i dati relativi alla Tabella 1.b sono in colore blu.\\
INSERIRE GRAFICO\\
\newline
Tramite regressione lineare troviamo la retta $h = A+B\theta$ che meglio interpreta i nostri dati. Le formule per il calcolo sono la \eqref{eq:a} e la \eqref{eq:b} riportate in appendice.
Verifichiamo la bontà di questo risultato tramite test del $\chi^2$. Ci viene un $\chi^2=boh$, troppo alto rispetto all'aspettativa di 9 gradi di libertà. L'incertezza è quindi troppo piccolo. Ricalcoliamo a posteriori l'incertezza e discutiamone la plausibilità.\\
\newline
Durante l'esperienza si è utilizzato un barometro in laboratorio per la misura della pressione atmosferica $P_A$ variabile nel tempo. Nel nostro manometro differenziale la pressione dovuta alla colonnina di acqua è data da: $\Delta P = \rho gh$. La pressione $P$ del nostro gas è quindi 
\begin{equation}
\label{eq:p}
P = P_A + \rho gh
\end{equation}
In appendice B è riportata la Tabella relativa alla pressione atmosferica misurata nel tempo, grazie a quella e all'equazione \eqref{eq:p} possiamo stimare la pressione $P$ del gas in funzione della temperatura.\\
Stima delle incertezze\\
\begin{table}[H]
	\centering
	\begin{tabular}{|c|c|} \hline
		\textbf{P {[pa]} } & \textbf{T {[\degree C]} }  \\ \hline
		130.5 & 2.306  \\ \hline
		130.5 & 2.306  \\ \hline
		130.5 & 2.306  \\ \hline
		130.5 & 2.306  \\ \hline
		130.5 & 2.306  \\ \hline
		130.5 & 2.306  \\ \hline
		130.5 & 2.306  \\ \hline
		130.5 & 2.306  \\ \hline
		130.5 & 2.306  \\ \hline
		130.5 & 2.306  \\ \hline
		130.5 & 2.306  \\ \hline
	\end{tabular}
	\caption{Pressione del gas in funzione della temperatura}
\end{table}

Da inserire in grafico :)

\subsection{Determinazione dello \emph{zero assoluto}}
blah blah blah solo conti da fare 

\section{Conclusione}
AHHHHH questa sarà da spararsi.....