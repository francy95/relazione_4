\section{Prima parte}
\subsection{Introduzione}
In questa prima parte di esperienza andremo a fare delle misurazioni sulla pressione del gas a volume costante e andando a variare la temperatura.
Decidiamo di partire da una temperatura circa uguale allo zero termico, e per questo riempiamo il comtenitore di acqua e ghiaccio: la coesistenza delle due fasi assicura infatti la temperatura richiesta.
Scegliamo quindi di salire fino a una temperatura di circa $20^{\degree} C$ per poi ridiscendere allo zero termico.
Per fare ciò, togliamo ghiaccio e aumentiamo acqua.
