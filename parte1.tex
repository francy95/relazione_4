\section{Introduzione}
\subsection{Preparazione dell'apparato sperimentale}

Per questa esperienza è d'obbligo prestare una particolare attenzione alla costruzione dell'apparato sperimentale, in quanto anche piccole dimenticanze o imprecisioni possono causare variazioni notevoli nelle misurazioni.
Si proceda quindi con la preparazione dell'apparato: versare un pò d'acqua nel contenitore, posarlo sopra l'agitatore magnetico ed inserire l'ancoretta magnetica al suo interno; controllare quindi che l'agitatore magnetico faccia vibrare l'ancoretta che, a sua volta, deve garantire la rotazione dell'acqua che la circonda.
Questa prima verifica è necessaria in quanto l'acqua diventerà il bagno termico per tutta la durata dell'esperimento e quindi il mescolamento dell'agitatore termico è una condizione necessaria per l'omogeneità dell'acqua che andrebbe, in sua assenza, a stratificarsi in base alla sua temperatura.
Controllare quindi che l'interno della bottiglia contenente il gas non sia in nessun modo bagnato e, tenendo aperti i rubinetti, fissarla stabilmente aòò'apposito sostegno in modo che essa possa restitere alla spinta idrostatica ed in modo da permettere il funzionamento dell'agitatore magnetico.
Collegare quindi un ramo del manometro ad uno dei rubinetti della bottiglia ed, aiutandosi con la siringa di plastica, versare acqua nel manometro tenendo i due rubinetti alla stessa altezza e prestando particolare attenzione in modo da evitare la formazione di bolle d'aria.
Riempire il manometro di acqua il più possibile, prestando attenzione che l'acqua raggiunga il più alto livello possibile non superando la parte rigida trasparente, grazie alla quale saremmo in grado di prendere le misure sulla scala millimetrata posta accanto.
Verificare nuovamente che la bottiglia contenente il gas sia perfettamente asciutta.

\subsection{Termalizzazione iniziale dell'aria}
Procedere quindi con la termalizzazione dell'aria.
Preparare il bagno termico intorno alla bottiglia e tramite il termometro controllare l'uniformità della temperatura, garantita dall'agitatore termico.
Scegliamo di termalizzare inizialmente intorno ai $0^{\degree}$ C per poi salire fino ad una temperatura di circa $20^{\degree}$ C e quindi ridiscendere nuovamente fino al valore iniziale.
Scegliendo questa procedura di misurazione la bottiglia si troverà sempre in sovrapressione rispetto alla pressione atmosferica: in questo caso quindi il manometro dovrà essere montato sul tavolo con il lato libero in grado di salire di circa 80 cm sopra la posizione iniziale (si consiglia quindi di scegliere un valore iniziale che non superi i 20 cm).
Scegliendo di termalizzare il sistema a $0^{\degree}$ C, la presenza di ghiaccio nel bagno dovrà essere presente in maniera predominante rispetto all'acqua: la coesistenza di queste due fasi assicura che la temperatura sia circa uguale allo zero termico.
Immergere quindi la bottiglia nel bagno termico fino al tappo, prestando attenzione che non raggiunga i giunti di connessione dei rubinetti: in questo modo permetteremo una migliore termalizzazione del gas.
Una volta raggiunto l'equilibrio termico alla temperatura iniziale prescelta, aprire il rubinetto e richiuderlo: in questo modo la pressione del gas all'interno della bottiglia viene uniformata con la pressione atmosferica $P_{A}$ ed il volume e le moli del gas rimangono costanti per tutta la durata dell'esperimento.
Procediamo poi con le misurazioni.\\

\subsection{Procedura di misurazione}
Come già affermato, scegliamo di termalizzare inizialmente intorno ai $0^{\degree}$ C per poi salire fino ad una temperatura di circa $20^{\degree}$ C e quindi ridiscendere nuovamente fino al valore iniziale.
Per aumentare la temperatura del bagno di fusione, procediamo sostituendo gradualmente l'acqua del recipiente con acqua più calda; al contrario, per diminuire la temperatura del bagno di fusione, procediamo sostituendo gradualmente l'acqua del recipiente con acqua di fusione.
Ad ogni variazione di temperatura bisogna riportare il valore del volume del gas al valore iniziale, anddando ad alzare o abbassare il ramo libero del manometro.
Andando a incrementare o diminuire la pressione della colonnina d'acqua che preme sul ramo libero del manometro, si va ad incrementare o diminuire la pressione del gas nell'ampolla che permetterà quindi al gas di tornare al suo volume iniziale.
Aspettare un tempo sufficiente per permettere ch eil gas nell'ampolla ed il bagno siano all'equilibrio termico e prendere quindi le misure di temperatura $T$ e di dislivello $\Delta h = h_1 - h_0$, dove $h_0$ è la posizione iniziale e $h_1$ è l'altezza a cui è stato alzato il ramo libero del manometro.
\\
Con questa procedura di misura le incertezze che abbiamo sono esclusivamente di risoluzione e per questo troviamo solamente una incertezza uguale per ogni misura, calcolata come:
\begin{equation}
\sigma (\Delta h) = \sqrt{(\sigma h_1)^2 + (\sigma h_0)^2} = \frac{\Delta X}{\sqrt{6}}
\end{equation}
dove con $\Delta X$ indichiamo il dislivello.
Troviamo quindi una incertezza sulle misurazioni di dislivello pari a $\sigma (\Delta h) = 0.04 m$ .
Per quando riguarda l'incertezza sulle misure di temperatura otteniamo invece una incertezza di risoluzione pari a $\sigma T = 0.1^{\degree}$ C.
Abbiamoscelto quindi di non considerare l'ultima cifra del termometro per varie motivazioni: prima tra tutte quella che questa cifra non era per niente stabile, ma continua ad oscillare tra più valori; la seconda è che questa cifra era relativa alla temperatura di un singolo punto, ovvero il punto in cui il termometro misurava la temperatura del bagno termico, mentre a noi interessava una misura più generale.
Per questo decidiamo di non considerare questa ultima cifra dell'ordine di $0.01^{\degree}$ C relativa alla misura di temperatura in quanto considerata non significativa e poco rilevante per questo esperimento.

\subsection{Pressione assoluta}
Durante tutta la durata dell'esperienza di laboratorio si deve tenere conto di una grandezza, la pressione atmosferica $P_A$, che non è costante nel tempo.
È quindi necessario tenere una traccia oraria relativa alle misurazioni effettuate ed amche prendere numerose misurazione dela pressione atmosferica nell'arco dell'esperienza.
Per effettuare queste misurazioni adoperiamo un barometro.
La pressione $P$ del gas nell'ampolla si ottiene:
\begin{equation}
P = P_A + \rho g h
\end{equation}
dove $P_A$ è la pressione atmosferica calcolata con il barometro, $\rho$ è la densità dell'acqua (pari a $1 \frac{kg}{dm^3}$), $g$
è l'accelerazione di gravità (pari a $9.807 \frac{m}{s^2}$).
L'incertezza su questa misura è data dalla risoluzione del barometro, ovvero $\sigma P = 2 mmHg$.

\section{Analisi dati: primo set}
\subsection{Pressione e temperatura}

\begin{table}[H]
\centering

	\begin{subtable}{.5\textwidth}
		\centering
		\begin{tabular}{|c|c|} \hline
			\textbf{$\Delta h {[cm]}$ } & \textbf{T {[\degree C]} }  \\ \hline
			0 & 0  \\ \hline
			8.5 & 2.3  \\ \hline
			15.9 & 4.1  \\ \hline
			24 & 6.1  \\ \hline
			31.7 & 8.1  \\ \hline
			40.4 & 10  \\ \hline
			49.8 & 12.1  \\ \hline
			56.8 & 14.3  \\ \hline
			63.3 & 16  \\ \hline
			71.8 & 18.4  \\ \hline
			77.6 & 20  \\ \hline
		\end{tabular}
		\caption{Aumento della temperatura}
	\end{subtable}%
	\begin{subtable}{.5\textwidth}
	\centering
	\begin{tabular}{|c|c|} \hline
		\textbf{$\Delta h {[cm]}$ } & \textbf{T {[\degree C]} }  \\ \hline
		77.6 & 20  \\ \hline
		73.4 & 18.8  \\ \hline
		66.2 & 17  \\ \hline
		59.4 & 15  \\ \hline
		52.4 & 13.1  \\ \hline
		44.2 & 11  \\ \hline
		36.2 & 8.9  \\ \hline
		26.5 & 6.9  \\ \hline
		18.9 & 5.1  \\ \hline
		-0.5 & 0  \\ \hline
	\end{tabular}
	\caption{Diminuzione della temperatura}
\end{subtable}

\caption{Dislivello $\Delta h$ in funzione della temperatura $T$}
\end{table}
In Tabella 1.a e Tabella 1.b troviamo riportati i dati raccolti durante la nostra esperienza rispettivamente aumentando e diminuendo la temperatura del bagno termico. 
Notiamo che l'ultimo valore della Tabella 1.a è uguale al primo valore della Tabella 1.b.
Troviamo graficati tali dati in Figura 1. 

\begin{figure}[H]
\centering
\includegraphics[width=\textwidth]{img/1}
\caption{Grafico di $\Delta h$ in funzione di $T$: scegliamo di rappresentare in colore rosso i dati riportati in Tabella 1.a ed in colore verde i dati riportati in Tabella 1.b}
\end{figure}

Tramite regressione lineare troviamo la retta $h = A+B\theta$ che meglio interpreta i nostri dati. 
Le formule per il calcolo sono la \eqref{eq:a} e la \eqref{eq:b} riportate in appendice.
Verifichiamo la bontà di questo risultato tramite test del $\chi^2$. Ci viene un $\chi^2=boh$, troppo alto rispetto all'aspettativa di 9 gradi di libertà.
L'incertezza è quindi troppo piccolo. 
Ricalcoliamo a posteriori l'incertezza e discutiamone la plausibilità.\\
\newline
Durante l'esperienza si è utilizzato un barometro in laboratorio per la misura della pressione atmosferica $P_A$ variabile nel tempo. Nel nostro manometro differenziale la pressione dovuta alla colonnina di acqua è data da: $\Delta P = \rho gh$. 
La pressione $P$ del nostro gas è quindi 
\begin{equation}
\label{eq:p}
P = P_A + \rho gh
\end{equation}
In appendice B è riportata la Tabella relativa alla pressione atmosferica misurata nel tempo, grazie a quella e all'equazione \eqref{eq:p} possiamo stimare la pressione $P$ del gas in funzione della temperatura.\\
Stima delle incertezze\\
\begin{table}[H]
	\centering
	\begin{tabular}{|c|c|} \hline
		\textbf{P {[Pa]} } & \textbf{T {[\degree C]} }  \\ \hline
		 &   \\ \hline
		 &   \\ \hline
		 &   \\ \hline
		 &   \\ \hline
		 &   \\ \hline
		 &   \\ \hline
		 &   \\ \hline
		 &   \\ \hline
		 &   \\ \hline
		 &   \\ \hline
		 &   \\ \hline
	\end{tabular}
	\caption{Pressione del gas $P$ in funzione della temperatura $T$}
\end{table}

Da inserire in grafico :)

\subsection{Determinazione dello \emph{zero assoluto}}
Lo zero assoluto è rappresentato graficamente dall'intersezione tra la retta del dislivello $\Delta h$ in funzione della temperatura $T$ e l'asse delle ascisse.
Questo punto rappresenta la condizione $P = 0$.
