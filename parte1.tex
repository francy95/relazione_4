\section{Prima parte}
\subsection{Introduzione}
\subsubsection{Preparazione dell'apparato sperimentale e strategie di misura}
Per questa esperienza è d'obbligo prestare una particolare attenzione per la costruzione dell'apparato sperimentale, in quanto anche piccole dimenticanze possono causare variazioni nelle misruazioni.
Si proceda quindi con la preparazione dell'apparato: inserire l'ancoretta magnetico all'interno del contenitore e, versando un pò d'acqua all'interno di esso e posandolo sopra all'agitatore termico, controllare che la rotazione dell'ancorett magnetica.

In questa prima parte di esperienza andremo a fare delle misurazioni sulla pressione del gas a volume costante e andando a variare la temperatura.
Decidiamo di partire da una temperatura circa uguale allo zero termico, e per questo riempiamo il comtenitore di acqua e ghiaccio: la coesistenza delle due fasi assicura infatti la temperatura richiesta.
Scegliamo quindi di salire fino a una temperatura di circa $20^{\degree} C$ per poi ridiscendere allo zero termico.
Per fare ciò, togliamo ghiaccio e aumentiamo acqua.
