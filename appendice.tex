\begin{appendices}

\section{Metodo dei minimi quadrati}
Applicando il metodo dei minimi quadrati alla legge lineare $y = A + Bx$ la discrepanza risulta:

\begin{equation}
\displaystyle\sum_{i=1}^{N}\frac{(y_i -A-Bx_i)^2}{(\sigma y_i)^2}
\end{equation}
\hspace{-1.8em}

Questa sommatoria è minimizzata da valori di A e B ottenuti come:

\begin{equation}
\label{eq:a}
A = \frac{(\sum_i w_i x_i^2)(\sum_i w_i y_i)-(\sum_i w_i x_i)(\sum_i w_i x_i y_i)}{\Delta} 
\end{equation}
e
\begin{equation}
\label{eq:b}
B = \frac{(\sum_i w_i)(\sum_i w_i y_i x_i)-(\sum_i w_i y_i)(\sum_i w_i x_i)}{\Delta}
\end{equation}

Dove abbiamo:

\[ \Delta = (\sum_i w_i)(\sum_i w_i x_i^2)-(\sum_i w_i x_i)^2  \quad \quad  w_i = \frac{1}{(\sigma y_i)^2}\]
\hspace{-1.8em}

Le incertezze sui parametri sono date da:

\begin{equation} 
\label{eq:sasb}
(\sigma A)^2 = \frac{\sum_i w_i x_i^2}{\Delta}\quad \quad (\sigma B)^2 = \frac{\sum_i w_i}{\Delta}
\end{equation}

\section{Pressione atmosferica}

\begin{table}[H]
	\centering
	\begin{tabular}{|c|c|} \hline
		\textbf{$P_A${[Pa]} } & \textbf{ORA {[hh:mm]} }  \\ \hline
		$9.716\times 10^4$ & 14.40  \\ \hline
		$9.176\times 10^4$ & 15.00  \\ \hline
		$9.769\times 10^4$ & 15.22  \\ \hline
		$9.769\times 10^4$ & 16.42  \\ \hline
		$9.772\times 10^4$ & 16.25  \\ \hline
		$9.759\times 10^4$ & 16.52  \\ \hline
	\end{tabular}
	\caption{Pressione atmosferica $P_A$ in funzione del tempo}
\end{table}




\section{Umidità dell'aria}

\begin{table}[H]
	\centering
	\begin{tabular}{|c|c|} \hline
		\textbf{Umidità} & \textbf{ORA {[hh:mm]} }  \\ \hline
		33.4\% & 14.00  \\ \hline
		35.6\% & 15.00  \\ \hline
		39.5\% & 16.00  \\ \hline
		42.1\% & 17.00  \\ \hline
		44.6\% & 18.00  \\ \hline
	\end{tabular}
	\caption{Umidità dell'aria in funzione del tempo}
\end{table}
\end{appendices}
