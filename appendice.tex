\begin{appendices}

\section{Metodo dei minimi quadrati}
Applicando il metodo dei minimi quadrati alla legge lineare $ y = A + Bx$ la discrepanza risulta:

\begin{equation}
\displaystyle\sum_{i=1}^{N}\frac{(y_i -A-Bx_i)^2}{(\sigma y_i)^2}
\end{equation}
\hspace{-1.8em}

Questa sommatoria è minimizzata da valori di A e B ottenuti come:

\begin{equation}
\label{eq:a}
A = \frac{(\sum_i w_i x_i^2)(\sum_i w_i y_i)-(\sum_i w_i x_i)(\sum_i w_i x_i y_i)}{\Delta} 
\end{equation}
e
\begin{equation}
\label{eq:b}
B = \frac{(\sum_i w_i)(\sum_i w_i y_i x_i)-(\sum_i w_i y_i)(\sum_i w_i x_i)}{\Delta}
\end{equation}
Dove abbiamo:
\[ \Delta = (\sum_i w_i)(\sum_i w_i x_i^2)-(\sum_i w_i x_i)^2  \quad \quad  w_i = \frac{1}{(\sigma y_i)^2}\]
\hspace{-1.8em}
Le incertezze sui parametri sono date da:

\begin{equation} 
\label{eq:sasb}
(\sigma A)^2 = \frac{\sum_i w_i x_i^2}{\Delta}\quad \quad (\sigma B)^2 = \frac{\sum_i w_i}{\Delta}
\end{equation}

\end{appendices}
