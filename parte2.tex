\section{Analisi dati: secondo set}
\subsection{Pressione e temperatura}

\begin{table}[H]
\centering

	\begin{subtable}{.5\textwidth}
		\centering
		\begin{tabular}{|c|c|} \hline
			\textbf{$\Delta h {[cm]}$ } & \textbf{T {[\degree C]} }  \\ \hline
			-0.5 & 0  \\ \hline
			18.3 & 4.7  \\ \hline
			30.4 & 7.77  \\ \hline
			43.4 & 11.5  \\ \hline
			57.9 & 15.1  \\ \hline
			69 & 18.1  \\ \hline
			76.6 & 20  \\ \hline
		\end{tabular}
		\caption{Aumento della temperatura }
	\end{subtable}%
	\begin{subtable}{.5\textwidth}
	\centering
	\begin{tabular}{|c|c|} \hline
		\textbf{$\Delta h {[cm]}$ } & \textbf{T {[\degree C]} }  \\ \hline
		76.6 & 20  \\ \hline
		58.7 & 15  \\ \hline
		49.6 & 12.5  \\ \hline
		35 & 8.7  \\ \hline
		19.9 & 5.1  \\ \hline
		-0.2 & 0.16  \\ \hline
	\end{tabular}
	\caption{Diminuzione della temperatura }
\end{subtable}

\caption{Dislivello $\Delta h$ in funzione della temperatura $T$ }
\end{table}
In Tabella 3.a e Tabella 3.b troviamo riportati il secondo set di dati raccolti durante la nostra esperienza rispettivamente aumentando e diminuendo la temperatura del bagno termico.
Troviamo graficati tali dati in Figura 3. 

