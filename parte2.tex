\section{Analisi dati: secondo set}
\subsection{Temperatura e dislivello}

\begin{table}[H]
\centering

	\begin{subtable}{.5\textwidth}
		\centering
		\begin{tabular}{|c|c|} \hline
			\textbf{$\Delta h {[cm]}$ } & \textbf{T {[\degree C]} }  \\ \hline
			-0.5 & 0  \\ \hline
			18.3 & 4.7  \\ \hline
			30.4 & 7.8  \\ \hline
			43.4 & 11.5  \\ \hline
			57.9 & 15.1  \\ \hline
			69.0 & 18.1  \\ \hline
			76.6 & 20.0  \\ \hline
		\end{tabular}
		\caption{Aumento della temperatura }
	\end{subtable}%
	\begin{subtable}{.5\textwidth}
	\centering
	\begin{tabular}{|c|c|} \hline
		\textbf{$\Delta h {[cm]}$ } & \textbf{T {[\degree C]} }  \\ \hline
		76.6 & 20.0  \\ \hline
		58.7 & 15.0  \\ \hline
		49.6 & 12.5  \\ \hline
		35.0 & 8.7  \\ \hline
		19.9 & 5.1  \\ \hline
		-0.2 & 0.2  \\ \hline
	\end{tabular}
	\caption{Diminuzione della temperatura }
\end{subtable}

\caption{Dislivello $\Delta h$ in funzione della temperatura $T$ }
\end{table}
In Tabella 2.a e Tabella 2.b troviamo riportati il secondo set di dati raccolti durante la nostra esperienza rispettivamente aumentando e diminuendo la temperatura del bagno termico.
Dividiamo nuovamente l'esperienza in due parti, divise dalla temperatura di rugiada 285 K.
\begin{figure}[H]
\centering
\includegraphics[width=\textwidth]{img/5}
\caption{Grafico di $\Delta h$ in funzione di $T$: 2 rette diverse per i dati divisi da 285 K}
\end{figure}

Iniziamo a valutare i dati al di sotto dei 285 K.\\
Procediamo nello stesso identico modo usato precedentemente e tramite regressione lineare troviamo la retta $h = A+BT$.
Otteniamo:
\[A = -10.70 \pm 0.03 m \quad  B = 0.0392 \pm 0.0001\frac{m}{K}\]
Otteniamo un $\chi^2$ pari a 244.96. 
Con i nostri gradi di libertà, questo valore va al di fuori di un intervallo di falso allarme del 10\% molto probabilmente perchè abbiamo sottostimato le incertezze sui valori.
Procediamo quindi calcolando le incertezze a posteriori tramite la relazione \eqref{eq:post} ed otteniamo una incertezza a posteriori pari a $ 8.7 $ mm. 
Per comprenderne il significato dobbiamo considerare la variazione di pressione atmosferica $P_A$ durante la sessione in cui sono state prese le misure. 
Questa analisi iniziale non tiene conto del suo effetto che è però già ben visibile in questa incertezza.\\
Le incertezze a posteriori di A e di B sono: $\sigma A = 0.2298m$ $\sigma B = 0.0008\frac{m}{K}$
\newline
Studiamo ora l'andamento dei dati al di sopra dei 285 K. 
Cerchiamo quindi analogamente la retta $h = A+BT$ che interpreta al meglio i nostri dati. 
Otteniamo:
\[A = -9.83 \pm 0.05 m \quad  B = 0.0361 \pm 0.0001\frac{m}{K}\]
Il test del $\chi^2$ ci fornisce la bontà di questo fit: $\chi^2 = 96.6174 $. 
Con i nostri gradi di libertà, questo valore va al di fuori di un intervallo di falso allarme del 10\%
Procediamo dunque a calcolare l'incertezza a posteriori con l'equazione \eqref{eq:post} ed otteniamo un incertezza a posteriori di $ 5.5 $ mm, che può essere comprensibile tenendo conto della risoluzioni di 1 millimetro del metro, della risoluzione del termometro trasferita e della variazione di pressione atmosferica durante l'acquisizione dei dati.\\
Le incertezze a posteriori di A e di B sono: $\sigma A = 0.2332m$ $\sigma B = 0.0008\frac{m}{K}$

\subsection{Determinazione dello \emph{zero assoluto}}
\begin{figure}[H]
\centering
\includegraphics[width=\textwidth]{img/6}
\caption{Grafico a barre della pressione $P$ in funzione della temperatura}
\end{figure}
In Figura 5 si nota il diverso comportamento della pressione al di sotto della temperatura di rugiada, per calcolare lo zero assoluto possiamo quindi utilizzare solamente i dati più affidabili, ovvero quelli al di sopra di tale temperatura.
Procediamo esattamente come nella sezione 4.2, utilizzando quindi le equazioni \eqref{eq:zero} e \eqref{eq:dzero} otteniamo:
\[T_0 = -3.8 \pm 1.9\, K \]